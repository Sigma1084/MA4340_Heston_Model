\subsection{Introduction}
\begin{frame}{Introduction to Heston Model}
	Improvement to Black Scholes Model.
	\vspace{0.5cm}

	Developed by Steven Heston in 1993.
	\begin{itemize}
		\item The stock price follows a general Brownian motion
		\item The stochastic process of the volatility is a CIR process
	\end{itemize}
	\vspace{0.5cm}

	Some important assumptions
	\begin{itemize}
		\item Interest Rate is constant
		\item No dividends
		\item European Style Options
		\item Frictionless Market
	\end{itemize}
\end{frame}

\begin{frame}{Stochastic Differential Equation}
	\begin{block}{Heston SDE}
		The Heston model is given by
		\begin{align*}
			dS_t &= \mu S_t dt + \sqrt{v_t} S_t dW_t^x & S_0 > 0 \\
			dv_t &= \kappa(\theta - v_t)dt + \sigma \sqrt{v_t} dW_t^v & v_0 > 0
		\end{align*}
		where \( W_t^x \) and \( W_t^v \) are Brownian motions with correlation
		\( \rho \) where \[ dW_t^x dW_t^v = \rho dt \quad |\rho| \leq 1 \]
	\end{block}
	\begin{itemize}
		\item \( \mu \) is the \textbf{drift}
		\item \( \kappa \) is the \textbf{speed of mean reversion}
		\item \( \theta \) is the \textbf{long term mean}
		\item \( \sigma \) is the \textbf{volatility coefficient}
		\item \( \rho \) is the \textbf{correlation coefficient}
	\end{itemize}
\end{frame}

\subsection{Garsinov Theorem}
\begin{frame}{Garsinov Theorem}
	The stock price and variance are under
	the historical measure $\mathbb{P}$. \\
	\vspace{0.5cm}

	By applying the Girsanov theorem one can find \\
	a probability measure $\mathbb{Q}$ such that we can represent \\
	the stock price in the form
	\begin{align*}
		dS_t &=  rS_t dt+ \sqrt{v_t}S_t d\Tilde{W}^x_t , S_0 > 0,\\
		dv_t &= \kappa( \theta - v_t)dt+ \sigma\sqrt{v_t}dW^v_t , v_0 > 0
		\label{eq:girsanov-heston-sde}
	\end{align*}
	where,
	\[
		\Tilde{W}^x_t = \left(W^x_t + \frac{\mu - r}{\sqrt{v_t}}\right)
	\]
\end{frame}

\subsection{Euler-Maruyama Method}
\begin{frame}{Euler-Maruyama Method}
	Numerical Method to solve Stochastic Differential Equations. \\
	Divide the interval \([0, T]\) into \( N \) equal sub-intervals
	of length \( \Delta t = \frac{T}{N} \)
	\[
		0 = t_0 < t_1 < \cdots < t_N = T
		\quad \quad \Delta t = \frac{T}{N} = t_{n+1} - t_n
	\]
	An SDE
	\[ \quad dX_t = r(t, X_t)dt + \sigma(t, W_t) dW_t \quad X_0 = 0 \]
	can be approximated by Euler-Maruyama method, given by
	\[ X_{n+1} = X_n + r(t_n, X_n) \Delta t + \sigma(t_n, X_n) \Delta W_n \]
	where \( \Delta W_n \sim \mathcal{N}\brak{0, \sqrt{\Delta t}} \)
\end{frame}

\subsection{Discretized Heston Model}
\begin{frame}{Discretized Heston Model}
	\begin{block}{Discretized Heston Model}
		The Heston model is given by
		\begin{align*}
			S_{t+1} &= S_t + rS_t \Delta t + \sqrt{v_t} Z_x \sqrt{\Delta t} \\
			v_{t+1} &= v_t + \kappa(\theta - v_t) \Delta t
				+ \sigma Z_v \sqrt{\Delta t}
		\end{align*}
		where \( Z_x, Z_v \sim \mathcal{N}\brak{0, 1} \)
		with correlation \( \rho \)
		So, we can write \( Z_x, Z_v \) as
		\begin{align*}
			Z_x &= \rho Z_1 + \sqrt{1 - \rho^2} Z_2 \\
			Z_v &= Z_1
		\end{align*}
		where \( Z_1, Z_2 \sim \mathcal{N}\brak{0, 1} \) are independent.
	\end{block}
\end{frame}


\subsection{Method of Moments}
\begin{frame}{Method of Moments}
	We define
	\[ Q_{t+1} \coloneqq \frac{S_{t+1}}{S_t} = 1 + r + \sqrt{v_t} Z_s \]
	Sample Moments of \( Q_{t+1} \) are given by
	\begin{align*}
		\Exp{Q_{t+1}} &= (1+r) \\
		\Exp{Q_{t+1}^2} &= (1+r)^2 + \Exp{v_t} \\
		\Exp{Q_{t+1}^3} &= (1+r)^3 + 3(1+r)\Exp{v_t} \\
		\Exp{Q_{t+1}^4} &= (1+r)^4 + 6(1+r)^2 \Exp{v_t} + 3\Exp{v_t^2} \\
		\Exp{Q_{t+1}^5} &= (1+r)^5 + 10(1+r)^3 \Exp{v_t} +
			15(1+r)\Exp{v_t^2} \\
	\end{align*}
\end{frame}

\subsection{Improvement of Heston Model}
\begin{frame}{Improvement of Heston Model}
	Rate of interest changes once every 3 months.
	\vspace{0.5cm}

	An improvement of the Heston model is given by
	a piecewise constant interest rate model.
	\vspace{0.5cm}

	Here, we assume that the interest rate is constant
	for 3 months and use CIR Model to estimate the new rate.
\end{frame}