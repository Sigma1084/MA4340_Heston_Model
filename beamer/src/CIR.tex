
\subsection{Rate Model}
\begin{frame}{Rate Models}
	\onslide<+->{
	Introduced by Oldrich Vasicek (1977) to model interest rates.
	\[ dr_t = \kappa(\theta - r_t)dt + \sigma dW_t \]
	where \( \kappa \) is the \textbf{mean reversion rate},
	\( \theta \) is the \textbf{long term mean} and
	\( \sigma \) is the \textbf{volatility} of the interest rate.
	}

	\onslide<+->{
	The model can be easily solved and the closed form solution is given by
	\[
		r_t \sim \mathcal{N} \brak{e^{-\kappa t} r_0 + \theta
		\brak{1 - e^{-\kappa t}}, \frac{\sigma^2}{2\kappa}
		\brak{1 - e^{-2\kappa t}}}
	\]
	The main disadvantage is that the interest rate can become negative.
	}
\end{frame}


\subsection{CIR Model}
\begin{frame}{CIR Model}
	The \textbf{CIR} model was introduced by John Carrington \textbf{Cox},
	Jonathan Edwards \textbf{Ingersoll} and Stephen Alan \textbf{Ross} (1985)
	and follows the following SDE\@.
	\[ dr_t = \kappa(\theta - r_t)dt + \sigma \sqrt{r_t} dW_t \] where
	\begin{itemize}
		\item \( \kappa \) is the \textbf{speed of mean reversion}
		\item \( \theta \) is the \textbf{long term mean}
		\item \( \sigma \) is the \textbf{volatility coefficient}
	\end{itemize}
	with initial rate \( r_0 \) given.
\end{frame}


\begin{frame}
	\begin{proposition}
		The exact solution of CIR is given by
		\[
			r_t = e^{-\kappa t} r_0 + \theta \brak{1 - e^{-\kappa t}} +
			\sigma e^{-\kappa t} \int_0^t e^{\kappa s} \sqrt{r_t} dW_s
		\]
	\end{proposition}
	\begin{proof}<+->
		\begin{align*}
%			\onslide<+->{
			dr_t &= \kappa(\theta - r_t)dt + \sigma \sqrt{r_t} dW_t \\
%			} \onslide<+->{
			e^{\kappa t} dr_t + \kappa e^{\kappa t} r_t dt
				&= \kappa\theta e^{\kappa t}dt +
				\sigma e^{\kappa t} \sqrt{r_t} dW_t \\
%			} \onslide<+->{
			\implies d\brak{e^{\kappa t} r_t} &=
				\kappa\theta e^{\kappa t}dt +
				\sigma e^{\kappa t} \sqrt{r_t} dW_t
				\quad \text{(Ito Doeblin Formula)} \\
%			} \onslide<+->{
			\implies \int_0^t d\brak{e^{\kappa s} r_s} &= \int_0^t
				\theta\kappa e^{\kappa s}ds +
				\int_0^t \sigma e^{\kappa s} \sqrt{r_s} dW_s \\
%			} \onslide<+->{
			\therefore\ r_t &= e^{-\kappa t} r_0 + \theta
				\brak{1 - e^{-\kappa t}} + \sigma
				\int_0^t e^{\kappa s} \sqrt{r_s} dW_s
%			}
		\end{align*}
	\end{proof}
\end{frame}


\begin{frame}
	\begin{proposition}
		The expectation of \( r_t \) is given by
		\[ \displaystyle \Exp{r_t} = e^{-\kappa t} r_0
		+ \theta \brak{1 - e^{-\kappa t}} \]
	\end{proposition}
	\begin{proof}<+->
		\begin{align*}
%			\onslide<+->{
			\Exp{r_t} &= \Exp{e^{-\kappa t} r_0 + \theta
				\brak{1 - e^{-\kappa t}} + \sigma
				\int_0^t e^{\kappa s} \sqrt{r_s} dW_s} \\
%			} \onslide<+->{
			&= e^{-\kappa t} r_0 + \theta\brak{1 - e^{-\kappa t}} +
				\Exp{\sigma \int_0^t e^{\kappa s} \sqrt{r_s} dW_s} \\
%			} \onslide<+->{
			\therefore \Exp{r_t} &= e^{-\kappa t} r_0 +
			\theta \brak{1 - e^{-\kappa t}}
%			}
		\end{align*}
	\end{proof}
%	\onslide<+->{
	Note that \( \displaystyle \lim_{t \to \infty} \Exp{r_t} = \theta \)
	(\textbf{long term mean}).
%	}
\end{frame}


\begin{frame}
	\begin{proposition}
		The variance of \( r_t \) is given by
		\[
			\Var{r_t} = \frac{\sigma^2}{\kappa} r_0
			\brak{e^{-\kappa t} - e^{-2\kappa t}} +
			\frac{\theta\sigma^2}{2\kappa} \brak{1 - e^{-\kappa t}}^2
		\]
	\end{proposition}
	\begin{proof}<+->
		\begin{align*}
%			\onslide<+->{
			\Var{r_t} &= \Exp{r_t^2} - \Exp{r_t}^2 \\
%			} \onslide<+->{
			&= \Exp{\brak{e^{-\kappa t} r_0 + \theta
				\brak{1 - e^{-\kappa t}} + \sigma e^{-\kappa t}
				\int_0^t e^{\kappa s} \sqrt{r_s} dW_s}^2} - \Exp{r_t}^2 \\
%			} \onslide<+->{
			&= \Exp{\brak{\Exp{r_t} + \sigma e^{-\kappa t}
				\int_0^t e^{\kappa s} \sqrt{r_s} dW_s}^2}
				- \Exp{r_t}^2 \\
%			} \onslide<+->{
			&= 2\Exp{\Exp{r_t} \sigma e^{-\kappa t}
				\int_0^t e^{\kappa s} \sqrt{r_s} dW_s}
				+ \Exp{\brak{\sigma e^{-\kappa t} \int_0^t e^{\kappa s}
				\sqrt{r_s} dW_s}^2}
%			}
		\end{align*}
	\end{proof}
\end{frame}

\begin{frame}
	\begin{proof}
		\begin{align*}
%			\onslide<+->{
			&= 2\ \Exp{r_t} e^{-\kappa t} \cdot 0 +
				\Exp{\brak{\sigma e^{-\kappa t} \int_0^t e^{\kappa s}
				\sqrt{r_s} dW_s}^2} \\
%			} \onslide<+->{
			&= \Exp{\sigma^2 e^{-2\kappa t} \int_0^t e^{2\kappa s} r_s ds}
				\quad (\text{Iso Isometry}) \\
%			} \onslide<+->{
			&= \sigma^2 e^{-2\kappa t}
				\int_0^t e^{2\kappa s} \Exp{r_s} ds \\
%			} \onslide<+->{
			&= \sigma^2 e^{-2\kappa t}
				\int_0^t e^{2\kappa s} \brak{e^{-\kappa s} r_0 +
				\theta \brak{1 - e^{-\kappa s}}} ds \\
%			} \onslide<+->{
			&= \frac{\sigma^2}{\kappa} e^{-2\kappa t} r_0
				\brak{e^{\kappa s} - 1}
				+ \theta\sigma^2 e^{-2\kappa t}
				\int_0^t \brak{e^{2\kappa s} - e^{\kappa s}} ds \\
%			} \onslide<+->{
			&= \frac{\sigma^2}{\kappa} r_0
				\brak{e^{-\kappa t} - e^{-2\kappa t}} +
				\frac{\theta\sigma^2}{2\kappa} \brak{1 - e^{-\kappa t}}^2
%			}
		\end{align*}
	\end{proof}
\end{frame}

\subsection{Method of Moments of Ito Integral}
\begin{frame}{Method of Moments of Ito Integral}
	\( \displaystyle I_t = \int_0^t \Delta_s dW_s \quad \) and
	\( \quad \Exp{I_t} = 0 \quad (\text{Zero Mean Property}) \)

	\( \displaystyle \Exp{I_t^2} = \Exp{\int_0^t \Delta^2(s) ds} \quad \)
	(Ito Isometry)

	\begin{proposition}
		\( \displaystyle \Exp{I_t^3} = 0 \)
	\end{proposition}

	\begin{proof}%<2->
		\( \displaystyle X_t^n = n \int_0^t X_s^{n-1} dX_s
		+ \frac{n(n-1)}{2} \int_0^t X_s^{n-2} ds \quad \)
		(Ito Doeblin Formula)
		\begin{align*}
			\Exp{\int_0^t I_s^2 \Delta(t) dW_t} &= 0
			\implies \Exp{I_t^3} = 0 + 3 \int_0^t \Exp{I_s} ds = 0
		\end{align*}
	\end{proof}
\end{frame}

\subsection{\( 3^{rd} \) Moment of \( r_t \)}
\begin{frame}{\( 3^{rd} \) Moment of \( r_t \)}
	\[ \Exp{r_t^3} = \Exp{\brak{A_t + B_t I_t}^3} \] where
	\( A_t = e^{-\kappa t} r_0 + \theta \brak{1 - e^{-\kappa t}} \quad \)
	\( B_t = \sigma e^{-\kappa t} \) and \\
	\( \displaystyle I_t = \int_0^t e^{\kappa s} \sqrt{r_s} dW_s \quad \)
	\begin{align*}
		\Exp{r_t^3} &= \Exp{\brak{A_t + B_t I_t}^3} \\
		&= A_t^3 + 3A_t^2 B_t\Exp{I_t} + 3A_t B_t^2\Exp{I_t^2}
			+ B_t^3\Exp{I_t^3} \\
		&= A_t^3 + 3A_t B_t^2\Exp{I_t^2} \\
		&= A_t \brak{A_t^2 + B_t^2 \int_0^t e^{2\kappa s} \Exp{r_s} ds } \\
		&= \Exp{r_t} \brak{\Exp{r_t}^2 + \Var{r_t}}
	\end{align*}
\end{frame}
