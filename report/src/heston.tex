The \textit{Heston Model} uses statistical methods to calculate and forecast
option pricing with the assumption that volatility is arbitrary.
The assumption that volatility is arbitrary, rather than constant,
is the key factor that makes \textit{stochastic volatility} models unique. \\

\noindent Key features of Heston Model:
\begin{itemize}
	\item It considers the possible correlation between a stock's price
	and its volatility.
	\item It does not require the stock prices to follow a lognormal
	probability distribution.(In the Black Scholes model we assume that
	stock prices follow a lognormal probability distribution).
\end{itemize}


\subsection{Application of Heston Model}

Developed by mathematician Steven Heston in 1993, the Heston model
was created to price options, which are a type of financial derivative.
Unlike other financial assets such as equities, the value of an option is
not based on the value of an asset but rather the change in an underlying
asset’s price. \\

In the Heston model, volatility is a mean reverting process.
Mean reversion means that the process does not wander off to
infinity but oscillates around a well-defined long term average value. \\

Each option is a contract between a buyer and seller, which gives the holder of
the option the right to buy or sell the underlying asset at a specific price.
All options have a specific expiration date, at which point the contract must
be executed at the previously set price or risk expiring. \\

However, the volatility of options depends on the price and maturity.
Therefore, the Heston model was designed to price an option while accounting
for these variations in market volatility. \\

There are two categories of options: calls and puts.
Calls allow the holder to buy at a specific price, and puts allow
the holder to sell at a specific price. \\

Once a call or put option has been purchased, the date at which
the holder can buy or sell depends on whether it is an American or
European option.
American options allow the holder to execute the option
anytime before the expiry date, while European options
only allow the holder to execute the option on the expiry date.
It’s important to note that the Heston model is only capable of
pricing European options.


\subsection{SDE}
\par Under the risk-neutral measure the Heston stochastic volatility model,
which is our point-of-departure, is specified by the following system of SDEs:
\begin{align}
	dS_t &= \mu S_t dt+ \sqrt{v_t}S_t dW^x_t , S_0 > 0 \\
	dv_t &= \kappa( \theta - v_t)dt+ \sigma\sqrt{v_t}dW^v_t , v_0 > 0
	\label{eq:sde-heston}
\end{align}
where,
\begin{enumerate}
	\item $\mu$ drift of the stock process(risk-free interest rate under
	risk-neutral measure)
	\item $\kappa$ mean reversion coefficient of the variance process
	\item $\theta$ long term mean of the variance process
	\item $\sigma$ volatility coefficient of the variance process
	\item $\rho$ correlation between $W^x_t$ and $dW^v_t$ i.e.,
\end{enumerate}

\begin{equation}
	dW^x_t dW^v_t = \rho dt
\end{equation}
with $|\rho_{x,\sigma} | < 1$.

The variance process, $v_t$, of the stock $S_t$ is a mean reverting
square root process, in which $\kappa > 0$ determines the speed of
adjustment of the volatility towards its theoretical mean,
$\theta > 0$, and $\sigma > 0$ is the second-order volatility, i.e.,
the variance of the volatility.


\subsection{Girsanov Theorem}
The stock price and variance in~\ref{eq:sde-heston} are under
the historical measure $\mathbb{P}$.
By applying the Girsanov theorem one can find a probability
measure $\mathbb{Q}$ such that we can represent the stock price in the form
\begin{align}
	dS_t &=  rS_t dt+ \sqrt{v_t}S_t d\Tilde{W}^x_t , S_0 > 0,\\
	dv_t &= \kappa( \theta - v_t)dt+ \sigma\sqrt{v_t}dW^v_t , v_0 > 0
	\label{eq:girsanov-heston-sde}
\end{align}
where,
\begin{equation}
	\Tilde{W}^x_t = \left(W^x_t + \frac{\mu - r}{\sqrt{v_t}}\right)
\end{equation}


\subsection{Euler - Maruyama Method}
Let us divide the time interval $[0,T]$ into $n$ parts
\begin{equation*}
	0 = t_0 < t_1 < t_2 < \cdots < t_n = T
\end{equation*}
We can choose equally spaced points $t_i$ such that
\begin{equation*}
	\Delta t = t_{i+1} - t_i = \frac{T}{N} \hspace{0.5cm}
	\forall \hspace{1mm}1\leq i \leq N
\end{equation*}
An SDE:
\begin{align*}
	dX_t &= r(t,X_t)dt + \sigma(t,X_t)dW_t \\
	X_0 &= 0
\end{align*}
can be approximated by:
\begin{align*}
	X_{i+1} &= X_i + r(t_i,X_i)\Delta t + \sigma(t_i,X_i)\Delta W_i \\
	X_0 &= 0
\end{align*}
where $X(t_i) = X_i$ , $W(t_i) = W_i$ and
$\Delta W_t \sim \mathcal{N} \brak{0,\sqrt{\Delta t}}$


\subsection{Discretized Heston Model}

The Heston process can be discretized using the Euler-Maruyama method.
Applying Euler-Maruyama method to Heston SDE~\ref{eq:girsanov-heston-sde}
we obtain:
\begin{align}
	S_{t+1} &= S_t + rS_t\Delta t + \sqrt{v_t}S_t Z_s\sqrt{\Delta t} \\
	v_{t+1} &= v_t + \kappa(\theta - v_t)\Delta t +
	\sigma \sqrt{v_t}Z_v\sqrt{\Delta t}
	\label{eq:discrete-heston}
\end{align}

where $Z_s,Z_v$ are standard normal random variables with correlation $\rho$.
So, we can write $Z_s,Z_v$ as,
\begin{align}
	Z_s &= \rho Z_1 + \sqrt{1-\rho^2}Z_2 \\
	Z_v &= Z_1
\end{align}


\subsection{Parameter Estimation using MOM}
In this method we equate the theoretical moments with the
sample moments from the data to obtain the parameters. \\

The $j^{th}$ moment of the random variable $Q_t$ is defined as $E(Q_t^j)$.
We use $M_j$ to denoted the $j^{th}$ sample moment obtained from the data. \\

The process to find the parameters is:
\begin{enumerate}
	\item Write m moments in terms of the m parameters that we are trying
	to estimate.

	\item Obtain sample moments from the data set.
	The $j^{th}$ sample moment denoted by $\hat{M_j}$ is obtained by
	raising each observation to the power of $j$ and taking the average
	of those terms.
	Symbolically,
	\begin{equation}
		\hat{M_j} = \frac{1}{n}\sum_{t=1}^{n}Q_{t+1}^j
	\end{equation}
	where,
	\begin{align}
		Q_{t+1} &= \frac{S_{t+1}}{S_{t}}\\
		&= 1 + r + \sqrt{v_t}Z_s
	\end{align}
	from~\ref{eq:discrete-heston}

	\item Substitute the $j^{th}$ sample moment for the $j^{th}$ moment
	in each of the m equations.
	That is, let $M_j = \hat{M_j}$.
	Now we have a system of m equations in m unknowns.

	\item Solve for each of the m parameters.
	The resulting parameter values are the method of moments estimates.
	We denote the method of moments estimate of a parameter $\alpha$ as
	$\hat{\alpha}_{MOM}$.

\end{enumerate}

When working with a data set of stock values, we are given values of $S_t$
rather than values of $Q_{t+1}$.
We can easily transform the data set into values of $Q_{t+1}$ by solving
for $\frac{S_{t+1}}{S_t}$ for each value of $t$.
We wish to write five moments of $Q_{t+1}$ in terms of the five
parameters $r, \kappa, \theta, \sigma,$ and $\rho$. \\

Letting $M_j$ represent the $j^{th}$ moment of $Q_{t+1}$,
we express formulas for the first moment, the second moment,
the fourth moment, and the fifth moment.
We have excluded the third moment because it does not add any
information to the system beyond the information available
from the first two moments. \\

\subsection{Calculating Theoretical Moments}

We know that
\[
	Q_{t+1} = 1 + r + \sqrt{v_t}\brak{\rho Z_1 + \sqrt{1 - \rho^2} Z_2}
\]
We need to find the moments of \( Q_{t+1}^n \) for \( 1 \leq n \leq 5 \).

\begin{align*}
	\Exp{Q_{t+1}} &= 1 + r + \Exp{\sqrt{v_t}\brak{\rho Z_1 +
		\sqrt{1 - \rho^2} Z_2}} \\
	&= 1 + r + \rho \Exp{\sqrt{v_t}} \Exp{Z_1} + \sqrt{1 - \rho^2}
		\Exp{\sqrt{v_t}}\Exp{Z_2} \\
	&= 1 + r + \rho \Exp{\sqrt{v_t}} \cdot 0 + \sqrt{1 - \rho^2}
		\Exp{\sqrt{v_t}} \cdot 0 \\
	&= 1 + r \numberthis
\end{align*}
We can observe that we just need to calculate for even powers of
\( \Exp{\brak{\sqrt{v_t}\brak{\rho Z_1 + \sqrt{1 - \rho^2} Z_2}}^n} \)
since expectations containing odd powers of \( Z \) is 0.
\begin{align*}
	&\Exp{\brak{\sqrt{v_t}\brak{\rho Z_1 + \sqrt{1 - \rho^2} Z_2}}^2} \\
	& = \rho^2 \Exp{v_t} \Exp{Z_1^2} + \brak{1 - \rho^2} \Exp{v_t} \Exp{Z_2^2}
		+ 2\Exp{v_t} \cdot 0 \\
	&= \Exp{v_t} + \Exp{v_t}\brak{\rho^2 - \rho^2} \\
	&= \Exp{v_t} \numberthis
\end{align*}

Using the same logic,
\begin{align*}
	&\Exp{\brak{\sqrt{v_t}\brak{\rho Z_1 + \sqrt{1 - \rho^2} Z_2}}^4} \\
	& = \rho^4 \Exp{v_t^2} \Exp{Z_1^4} + \brak{1 - \rho^2}^2
		\Exp{v_t^2} \Exp{Z_2^4} + \\
	& \quad + 6\Exp{v_t^2} \rho^2 \brak{1 - \rho^2} \Exp{Z_1^2} \Exp{Z_2^2} \\
	&= \Exp{v_t^2} \brak{3\rho^4 + 3\brak{1 - \rho^2}^2
		+ 6\rho^2\brak{1-\rho^2}} \\
	&= \Exp{v_t^2} \brak{6\rho^4 - 6\rho^2 + 3 - 6\rho^4 + 6\rho^2} \\
	&= 3\Exp{v_t^2} \numberthis
\end{align*}

Now, we can use the above results to calculate the moments of \( Q_{t+1} \). \\
Let's say
\[
	B_n \coloneqq\Exp{\brak{\sqrt{v_t}\brak{\rho Z_1 +
	\sqrt{1 - \rho^2} Z_2}}^n} \numberthis
\]

\begin{align*}
	\Exp{Q_{t+1}^2} &= (1+r)^2 + 2(1+r)B_1 + B_2 \\
	&= (1+r)^2 + \Exp{v_t} \numberthis \\
	\\
	\Exp{Q_{t+1}^3} &= (1+r)^3 + 3(1+r)^2 B_1 + 3(1+r)B_2 + B_3 \\
	&= (1+r)^3 + 3(1+r)\Exp{v_t} \numberthis \\
	\\
	\Exp{Q_{t+1}^4} &= (1+r)^4 + 6(1+r)^2 B_2 + B_4 \\
	&= (1+r)^4 + 6(1+r)^2 \Exp{v_t} + 3\Exp{v_t^2} \numberthis \\
	\\
	\Exp{Q_{t+1}^5} &= (1+r)^5 + 10(1+r)^3 B_2 + 5(1+r)B_4 \\
	&= (1+r)^5 + 10(1+r)^3 \Exp{v_t} + 15(1+r)\Exp{v_t^2} \numberthis \\
\end{align*}


\subsection{Expectation of $j^{th}$ moment}

Moments of $Q_{t+1}$

Recall the discretized formula for $Q_{t+1}$.

The third moment of $Q_{t+1}$ uses the same two parameters that the first and second moments use, so the third moment will not be used when solving for the method of moments parameter estimates.



