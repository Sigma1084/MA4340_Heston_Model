\begin{itemize}
	\item \textbf{Options} are financial derivatives that give buyers the
		right, but not the obligation, to buy or sell an underlying asset
		at an agreed-upon price and date.
	\item A \textbf{call option} gives the holder the right, but not the
		obligation, to buy the underlying security at the strike price on or
		before expiration.
	\item A \textbf{put option} gives the holder the right, but not the
		obligation, to sell the underlying stock at the strike price on or
		before expiration.
    \item \textbf{American options} can be exercised at any time between the
		date of purchase and the expiration date.
    \item \textbf{European options} can only be exercised at the end of
		their lives on their expiration date.
    \item \textbf{Volatility} represents how large an asset's prices swing
		around the mean price—it is a statistical measure of its
		dispersion of returns.
		Volatile assets are often considered riskier than less volatile
		assets because the price is expected to be less predictable.
    \item \textbf{Market risk} is the possibility that an individual or other
		entity will experience losses due to factors that affect the
		overall performance of investments.
		Market risk may arise due to changes to interest rates,
		exchange rates, geopolitical events, or recessions.
    \item \textbf{Mean reversion} is a theory used in finance that suggests
		that asset price volatility and historical returns eventually will
		revert to the long-run mean or average level of the entire dataset. \\
		Credit : Investopedia
\end{itemize}